\documentclass[10pt]{article}
\usepackage[english]{babel}
\usepackage[utf8]{inputenc}
\usepackage{palatino,color,graphicx}
\usepackage{float}
\usepackage{amsmath}
\usepackage{caption}
\usepackage{subcaption}
\usepackage{fullpage}
\usepackage{verbatim}

\setlength{\parskip}{1pc}
\setlength{\parindent}{0pt}
\setlength{\topmargin}{-3pc}
\setlength{\textheight}{9.5in}
\setlength{\oddsidemargin}{0pc}
\setlength{\evensidemargin}{0pc}
\setlength{\textwidth}{6.5in}

\title{Summary of Meeting July 19, 2016}
\author{Putu Ayu Sudyanti}
\date{}

\begin{document}
\maketitle
The current algorithm poses some mixing problem in which the chain failed to explore other modes in the posterior. To fix this, an additional Metropolis-Hastings step is needed to guarantee that other possible solutions are considered. There are 2 different ways in which we can attempt this procedure; with independent or dependent transition probability. 
\begin{enumerate}
\item Independent transition probability \\
The first method assumes that sampling of $(\theta_i^n \theta_i^t)$ and $\zeta^i$ are done independently in the transition probability of the Metropolis-Hastings step. The algorithm can be written as follows: 
\begin{itemize}
\item[i.] Randomly sample $\theta_i^* = (\theta_i^{n^*}, \theta_i^{t^*})$ uniformly each with probability $\dfrac{1}{9}$
\item[ii.] Pick $P(\zeta_i^*|\theta_i^*)$ based on the following:
\begin{itemize}
\item If $\zeta_i$ is not a singleton:\\
$\zeta_i^* = \zeta_{c^*}$ with probability $\dfrac{n_{c^*}}{N-1}$ where $c^* = 1,2, \cdots, K$
\item Otherwise: \\
$\zeta_i^* \sim H $ with probability 1
\end{itemize}
\item[iii.] Determine the acceptance probability of the Metropolis-Hastings by: \\
\\
$\begin{array}{rcl}
r & = & \dfrac{
P(\zeta_i^*, \theta_i^*|y) \times q(\zeta_i, \theta_i|\zeta_i^*, \theta_i^*)
}{
P(\zeta_i, \theta_i|y) \times q(\zeta_i^*, \theta_i^{n*}, \theta_i^{t*}|\zeta_i, \theta_i^{n}, \theta_i^{t})
}\\
\noalign{\medskip}
&= &\dfrac{
	\dfrac{1}{9} \dfrac{n_{c*}}{n+\alpha-1} \times P(y_i|\zeta_i^*, \theta_i^*)
	}{
	\dfrac{1}{9}\dfrac{n_{c}}{n+\alpha-1}\times P(y_i|\zeta_i, \theta_i)
	} 
	\dfrac{\dfrac{1}{9}\dfrac{n_{c}}{N-1}}{\dfrac{1}{9}\dfrac{n_{c^*}}{N-1}
	}\\
\noalign{\bigskip}
&= &\dfrac{P(y_i|\zeta_i^*, \theta_i^*)}{P(y_i|\zeta_i, \theta_i)}
\end{array}$
\end{itemize}
\item Dependent transition \\
This second method assumes that the choice of $\zeta_i$ should depend on the sample of $(\theta_i^n, \theta_i^t)$ in the transition probability of the algorithm. The pseudo-code is as follows:
\begin{itemize}
\item[i.] Propose $\theta^* = (\theta_i^{n*}, \theta_i^{t*})$ uniformly with probability $\dfrac{1}{9}$
\item[ii.] For nonsingleton: \\
Set $\zeta^* = \zeta_{c^*}$ for an existing cluster $c^* = 1, 2, \cdots, K$ with probability \\
\\
$\begin{array}{rcl}
q(\zeta^* = \zeta_{c^*}|\theta^*) \propto \dfrac{
	n_{c^*} P(y_i|\zeta_{c^*}, \theta^*)
}{z(\theta^*)}
\end{array}$ \\
where $z(\theta^*) = \sum_{j=1}^K n_j P(y_i|\zeta_j, \theta^*)$. The acceptance rate will be: \\
$\begin{array}{rcl}
r & = & \dfrac{
P(y_i, \theta^*, \zeta^*)
}{
P(y_i, \theta, \zeta)
} \times \dfrac{
q(\theta, \zeta)
}{
q(\theta^*, \zeta^*)
}\\
\noalign{\medskip}

&=& \dfrac{
	\dfrac{1}{9} \dfrac{
	n_{c^*}
	}{
	N+\alpha-1
	} P(y_i|\theta^*, \zeta_{c^*})
}{
\dfrac{1}{9} \dfrac{
n_{c}
}{
N+\alpha-1
} P(y_i|\theta, \zeta)} \times 
\dfrac{\dfrac{1}{9} 
\dfrac{
n_{c} P(y_i|\zeta_{c}, \theta)
}{
z(\theta)}
}{\dfrac{1}{9} 
\dfrac{n_{c^*} P(y_i|\zeta_{c^*}, \theta^*)
}{z(\theta^*)}
}\\
\noalign{\medskip}
&=& \dfrac{z(\theta^*)}{z(\theta)}
\end{array}$\\
\item[iii.] For singleton: \\
The transition probability is $q(\zeta^*) = p(\zeta^*)$ then the acceptance rate becomes: \\

$\begin{array}{rcl}
r & = & \dfrac{
P(y_i, \theta^*, \zeta^*)
}{
P(y_i, \theta, \zeta)
} \times \dfrac{
q(\theta, \zeta)
}{
q(\theta^*, \zeta^*)
}\\
\noalign{\medskip}

&=& \dfrac{
	\dfrac{1}{9} P(\zeta^*) P(y_i|\theta^*, \zeta_{c^*})
}{
\dfrac{1}{9} P(\zeta) P(y_i|\theta, \zeta)
} \times 
\dfrac{
\dfrac{1}{9} P(\zeta)
}{
\dfrac{1}{9} P(\zeta^*)
}\\
\noalign{\medskip}
&=& \dfrac{P(y_i|\zeta^*, \theta_*)}{P(y_i|\zeta, \theta)}
\end{array}$\\
\end{itemize}  
\end{enumerate}
\end{document}